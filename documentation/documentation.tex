\documentclass[10pt,a4paper]{article}
%
%-------------------------packages--------------------------------------
%
\usepackage[utf8x]{inputenc}
\usepackage{a4wide}
\usepackage{graphicx}
\usepackage{rotating}
\usepackage{natbib}
\usepackage{afterpage}
\usepackage{lscape}
\usepackage{color}
\usepackage{amsmath}
\usepackage{amssymb}
\usepackage{times,mathptmx}
\usepackage{appendix}
\usepackage{subfigure}
\usepackage{tabularx}
\usepackage{ltablex}
\usepackage{multicol}
\usepackage{captcont}  % for caption continuation ; appendix
%\usepackage{fixmath}   % for \mathbold{}  warning: {\rm \theta} or {\rm \phi}
%is wrong
\usepackage{float}
\usepackage{longtable}
\usepackage{ltcaption}
\usepackage{epsfig,psfrag,pdflscape}
\usepackage[varg]{txfonts} %varg to distinguish between v and nu
\usepackage{calrsfs}
\usepackage[breaklinks]{hyperref} %for line breaks in list of figures
\usepackage{thumbpdf}
\usepackage[bottom,hang,stable]{footmisc}
\usepackage{algorithm}
\usepackage{algorithmic}
%the second language defines main language used for contents and bibliography
\usepackage[ngerman,english]{babel}
\usepackage[T1]{fontenc}


\usepackage[tableposition=t]{caption}
%
%----------------------hypersetup-------------------------------------------------
%
\newcommand{\notesTitle}{Documentation}
\newcommand{\notesAuthor}{Levin Hennicker}
%
\hypersetup{
   pdfpagemode = {UseNone}, % Verhindert die automatische Anzeige der Lesezeichen
   pdftitle = {\notesTitle},
   pdfauthor = {\notesAuthor}
}
%
%----------------------page layout-------------------------------------------------
%
% Seitenlayout
\usepackage{geometry}
\geometry{
  left=2cm,
  right=2cm,
  top=2cm,
  bottom=2cm,
  bindingoffset=5mm
}
%
%---------------------------definitions---------------------------------
%
\definecolor{blue}{rgb}{0.,0.,0.5}
\definecolor{gray}{rgb}{0.,0.5,0.}
\definecolor{rrr}{rgb}{0.5,0.,0.}
\definecolor{darkblue}{rgb}{0,0,0.5}
\definecolor{orange}{rgb}{1.,0.5,0.}
\definecolor{darkgreen}{rgb}{0,0.5,0}
%
\newcommand{\defeq}{\mathrel{\mathop:}=}
\newcommand{\einruecken}{\noindent\hspace*{10mm}}
\newcommand{\einrueckenb}{\noindent\hspace*{17mm}}

\newcommand{\specialcell}[2][c]{%
  \begin{tabular}[#1]{@{}l@{}}#2\end{tabular}}
%
\newcommand{\ih}{\ensuremath{I_{\rm H}}}
\newcommand{\nh}{\ensuremath{n_{\rm H}}}
\newcommand{\mh}{\ensuremath{m_{\rm p}}}
\newcommand{\yh}{\ensuremath{Y_{\rm H}}}
\newcommand{\ihe}{\ensuremath{I_{\rm He}}}
\newcommand{\nhe}{\ensuremath{n_{\rm He}}}
\newcommand{\yhe}{\ensuremath{Y_{\rm He}}}
\newcommand{\mhe}{\ensuremath{m_{\rm He}}}
\newcommand{\iii}{\ensuremath{I_{\rm i}}}
\newcommand{\nii}{\ensuremath{n_{\rm i}}}
\newcommand{\mii}{\ensuremath{m_{\rm i}}}
\newcommand{\aii}{\ensuremath{A_{\rm i}}}
\newcommand{\zii}{\ensuremath{Z_{\rm i}}}
\newcommand{\yii}{\ensuremath{Y_{\rm i}}}
\newcommand{\nkk}{\ensuremath{n_{\rm k}}}
\newcommand{\mkk}{\ensuremath{m_{\rm k}}}
\newcommand{\ykk}{\ensuremath{Y_{\rm k}}}

% This is for txfonts: it sets the ring lower, touching the A.
\newcommand{\msun}{\ensuremath{M_{\odot}}}
\newcommand{\lsun}{\ensuremath{L_{\odot}}}
\newcommand{\rsun}{\ensuremath{R_{\odot}}}
\newcommand{\Zsun}{\ensuremath{Z_{\odot}}}
\newcommand{\Teff}{\ensuremath{T_{\rm eff}}}
\newcommand{\vinf}{\ensuremath{v_{\infty}}}
\newcommand{\mdot}{\ensuremath{\dot{M}}}
\newcommand{\Ndot}{\ensuremath{N_{\odot}}}

\newcommand{\ratio}{\ensuremath{v_{\infty}/v_{\rm esc}}}

\newcommand{\msunyr}{\ensuremath{M_{\odot} {\rm yr}^{-1}}}
\newcommand{\Mdu}{\ensuremath{\cdot 10^{-6}\,M_{\odot} {\rm yr}^{-1}}}
\newcommand{\mdu}{\ensuremath{10^{-6}\,M_{\odot} {\rm yr}^{-1}}}

\newcommand{\beq}{\begin{equation}}
\newcommand{\eeq}{\end{equation}}
\newcommand{\beqa}{\begin{eqnarray}}
\newcommand{\eeqa}{\end{eqnarray}}

\newcommand{\nbeq}{\begin{equation*}}
\newcommand{\neeq}{\end{equation*}}

\newcommand{\eu}{\ensuremath{{\rm e}}}

\newcommand{\kms}{\ensuremath{{\rm km}\,{\rm s}^{-1}}}
\newcommand{\gcmc}{\ensuremath{{\rm g}\,{\rm cm}^{-3}}}
\newcommand{\seconds}{\ensuremath{{\rm s}}}
\newcommand{\magn}{\ensuremath{^{\rm m}}}
\newcommand{\half}{\ensuremath{\frac{1}{2}}}

\newcommand{\rarrow}{\rightarrow}
\newcommand{\dd}{{\rm d}}

\newcommand{\HI}  {H\,{\sc i}}
\newcommand{\HII} {H\,{\sc ii}}
\newcommand{\HeI} {He\,{\sc i}}
\newcommand{\HeII}{He\,{\sc ii}}
\newcommand{\HeIII}{He\,{\sc iii}}

\newcommand{\CII}{C\,{\sc ii}}
\newcommand{\CIII}{C\,{\sc iii}}
\newcommand{\CIV}{C\,{\sc iv}}
\newcommand{\CV}{C\,{\sc v}}

\newcommand\NII{N\,{\sc ii}}
\newcommand\NIII{N\,{\sc iii}}
\newcommand\NIV{N\,{\sc iv}}
\newcommand\NV{N\,{\sc v}}
\newcommand\NVI{N\,{\sc vi}}
\newcommand\NVII{N\,{\sc vii}}

\newcommand\OI{O\,{\sc i}}
\newcommand\OII{O\,{\sc ii}}
\newcommand\OIII{O\,{\sc iii}}
\newcommand\OIV{O\,{\sc iv}}
\newcommand\OV{O\,{\sc v}}
\newcommand\OVI{O\,{\sc vi}}
\newcommand\OVII{O\,{\sc vii}}

\newcommand\NeII{Ne\,{\sc ii}}

\newcommand\SiII{Si\,{\sc ii}}
\newcommand\SiIII{Si\,{\sc iii}}
\newcommand\SiIV{Si\,{\sc iv}}
\newcommand\SiV{Si\,{\sc v}}
\newcommand\SiVI{Si\,{\sc vi}}

%\newcomand{\SVI}{S\,{\sc vi}}

\newcommand{\PIV}{P\,{\sc iv}}
\newcommand{\PV}{P\,{\sc v}}
\newcommand{\PVI}{P\,{\sc vi}}
\newcommand{\PVII}{P\,{\sc vii}}

\newcommand{\FeII}{Fe\,{\sc ii}}
\newcommand{\FeIII}{Fe\,{\sc iii}}
\newcommand{\FeIV}{Fe\,{\sc iv}}

\newcommand{\Hep}{H$_{\epsilon}$}
\newcommand{\Hd} {H$_{\rm \delta}$}
\newcommand{\Hg} {H$_{\rm \gamma}$}
\newcommand{\Hb} {H$_{\rm \beta}$}
\newcommand{\Ha} {H$_{\rm \alpha}$}
\newcommand{\Lya}{L$_{\rm \alpha}$}
\newcommand{\Bra}{Br$_{\rm \alpha}$}
\newcommand{\Brg}{Br$_{\rm \gamma}$}
\newcommand{\Pfg}{Pf$_{\rm \gamma}$}
\newcommand{\Pfb}{Pf$_{\rm \beta}$}

\newcommand{\Lum}{\ensuremath{\log L}}
\newcommand{\LL}{\ensuremath{(L/L_{\odot})}}
\newcommand{\logLL}{\ensuremath{\log (L/L_{\odot})}}
\newcommand{\MV}{\ensuremath{M_V}}

\newcommand{\Rstar}{\ensuremath{R_{\ast}}}
\newcommand{\Rmax}{\ensuremath{R_{\rm max}}}
\newcommand{\Rmin}{\ensuremath{R_{\rm min}}}
\newcommand{\Mstar}{\ensuremath{M_{\ast}}}
\newcommand{\Lstar}{\ensuremath{L_{\ast}}}

\newcommand{\Mwm}{\ensuremath{\log (modWMR)}}
\newcommand{\Dmom}{\ensuremath{D_{\rm mom}}}

\newcommand{\logg}{\ensuremath{\log g}}
\newcommand{\YHe}{\ensuremath{Y_{\rm He}}}
\newcommand{\gc}{\ensuremath{g_{\rm cent}}}

\newcommand{\Te}{\ensuremath{T_{\rm e}}}
\newcommand{\Trad}{\ensuremath{T_{\rm rad}}}
\newcommand{\Tgas}{\ensuremath{T_{\rm gas}}}
\newcommand{\Tshock}{\ensuremath{T_{\rm s}}}
\newcommand{\Tshockmax}{\ensuremath{T_{\rm s}^{\infty}}}

\newcommand{\vesc}{\ensuremath{v_{\rm esc}}}
\newcommand{\vth}{\ensuremath{v_{\rm th}}}
\newcommand{\vrad}{\ensuremath{v_{\rm rad}}}
\newcommand{\vrot}{\ensuremath{v_{\rm rot}}}
\newcommand{\vcrit}{\ensuremath{v_{\rm crit}}}
\newcommand{\vsini}{\ensuremath{v{\thinspace}\sin{\thinspace}i}}
\newcommand{\vturb}{\ensuremath{v_{\rm turb}}}
\newcommand{\vmic}{\ensuremath{v_{\rm mic}}}
\newcommand{\vmac}{\ensuremath{v_{\rm mac}}}
\newcommand{\vsound}{\ensuremath{v_{\rm s}}}

\newcommand{\mx}{\ensuremath{m_{\rm x}}}

\newcommand{\adiabsound}{\ensuremath{a_{\rm s}}}

\newcommand{\dvdr}{\ensuremath{\dd v/\dd r}}
\newcommand{\fcl}{\ensuremath{f_{\rm cl}}}
\newcommand{\fv}{\ensuremath{f_{\rm X}}}
\newcommand{\xne}{\ensuremath{n_{\rm e}}}
\newcommand{\taur}{\ensuremath{\tau_{\rm Ross}}}

\newcommand{\elecden}{\ensuremath{n_{\rm e}}}
\newcommand{\protden}{\ensuremath{n_{\rm p}}}

\newcommand{\nl}{\ensuremath{n_{\rm l}}}
\newcommand{\nup}{\ensuremath{n_{\rm u}}}
\newcommand{\nuul}{\ensuremath{\nu_{\rm ul}}}
\newcommand{\taus}{\ensuremath{\tau_{\rm Sob}}}

\newcommand{\lxray}{\ensuremath{L_{\rm x}}}
\newcommand{\lbol}{\ensuremath{L_{\rm bol}}}

\newcommand{\grad}{\ensuremath{g_{\rm rad}}}

\newcommand{\trip}{\ensuremath{\lambda\lambda4634-4640-4642}}
\newcommand{\qua}{\ensuremath{\lambda\lambda4510-4514-4518}}

\newcommand{\nivem}{\ensuremath{\lambda4058}}
\newcommand{\nivab}{\ensuremath{\lambda6380}}
\newcommand{\niiir}{\ensuremath{\lambda4640}}

\newcommand{\Jbar}{\ensuremath{\bar J}}
\newcommand{\Ibar}{\ensuremath{\bar I}}

\newcommand\ie{\hbox{i.e.,}}
\newcommand\eg{\hbox{e.g.,}}


%from levin:
%%\newcommand{\vecown}[1]{\ensuremath{\underline{#1}}}
%%\newcommand{\matown}[1]{\ensuremath{\underline{\underline{#1}}}}
%\newcommand{\vecown}[1]{\ensuremath{\vec{#1}}}
%\newcommand{\matown}[1]{\ensuremath{\vec{#1}}}
\newcommand{\vecown}[1]{\ensuremath{\boldsymbol{#1}}}
\newcommand{\matown}[1]{\ensuremath{\boldsymbol{#1}}}

\newcommand{\iincv}{\ensuremath{\vecown{I}_{\rm inc}}}
\newcommand{\scontv}{\ensuremath{\vecown{S}^{\rm (C)}}}
\newcommand{\slinev}{\ensuremath{\vecown{S}^{\rm (L)}}}

%\newcommand{\alo}{\ensuremath{\Lambda^\ast}}
\newcommand{\alo}{\ensuremath{\Lambda^{\rm (A)}}}
\newcommand{\alom}{\ensuremath{\matown{\Lambda}^\ast}}
\newcommand{\unitym}{\ensuremath{\matown{1}}}


\newcommand{\icore}{\ensuremath{I_{\rm c}}}
\newcommand{\epsc}{\ensuremath{\epsilon_{\rm C}}}
\newcommand{\epsl}{\ensuremath{\epsilon_{\rm L}}}
\newcommand{\scont}{\ensuremath{S_{\rm C}}}
\newcommand{\sline}{\ensuremath{S_{\rm L}}}
\newcommand{\kcont}{\ensuremath{k_{\rm C}}}
\newcommand{\kline}{\ensuremath{k_{\rm L}}}
\newcommand{\kapline}{\ensuremath{\kappa_{\rm 0}}}
\newcommand{\kappae}{\ensuremath{\kappa_{\rm e}}}
\newcommand{\kappal}{\ensuremath{\kappa_{\rm L}}}
\newcommand{\chibar}{\ensuremath{\bar{\chi}}}
\newcommand{\chibarl}{\ensuremath{\bar{\chi}_{\rm L}}}
\newcommand{\chic}{\ensuremath{\chi_{\rm C}}}
\newcommand{\chil}{\ensuremath{\chi_{\rm L}}}
\newcommand{\chith}{\ensuremath{\chi_{\rm Th}}}
\newcommand{\chiinf}{\ensuremath{\chi_{\rm \infty}}}
\newcommand{\const}{\ensuremath{{\rm const.}}}
\newcommand{\vvinf}{\ensuremath{V_{\infty}}}
\newcommand{\pihalf}{\ensuremath{\frac{\pi}{2}}}
\newcommand{\tpihalf}{\ensuremath{\frac{3\pi}{2}}}
\newcommand{\profile}{\ensuremath{\Phi_x}}
\newcommand{\profilenu}{\ensuremath{\Phi_\nu}}
\newcommand{\nx}{\ensuremath{N_{\rm x}}}
\newcommand{\ny}{\ensuremath{N_{\rm y}}}
\newcommand{\nz}{\ensuremath{N_{\rm z}}}
\newcommand{\nnz}{\ensuremath{N_{\rm NZ}}}
\newcommand{\nnu}{\ensuremath{N_{\rm F}}}
\newcommand{\nnue}{\ensuremath{N_{\rm \nu}}}
\newcommand{\ntheta}{\ensuremath{N_{\rm \theta}}}
\newcommand{\nphi}{\ensuremath{N_{\rm \phi}}}
\newcommand{\nomega}{\ensuremath{N_{\rm \Omega}}}
\newcommand{\ncpu}{\ensuremath{N_{\rm CPU}}}
\newcommand{\vmin}{\ensuremath{v_{\rm min}}}
\newcommand{\hei}{\ensuremath{I_{\rm He}}}
\newcommand{\mhyd}{\ensuremath{m_{\rm H}}}
\newcommand{\xobs}{\ensuremath{x_{\rm obs}}}
\newcommand{\xcmf}{\ensuremath{x_{\rm cmf}}}

\newcommand{\indx}[1]{\ensuremath{{#1}}}
\newcommand{\indxijk}{\ensuremath{\indx{ijk}}}
\newcommand{\scontijk}{\ensuremath{S_{\indx{ijk}}^{\rm (C)}}}
\newcommand{\slineijk}{\ensuremath{S_{\indx{ijk}}^{\rm (L)}}}
\newcommand{\chicijk}{\ensuremath{\chi_{\indx{ijk}}^{(\rm C)}}}
\newcommand{\chibarijk}{\ensuremath{\bar{\chi}_{\indx{ijk}}^{\rm (L)}}}
\newcommand{\indxu}{\ensuremath{{\rm u}}}
\newcommand{\indxp}{\ensuremath{{\rm p}}}
\newcommand{\indxd}{\ensuremath{{\rm d}}}
\newcommand{\indxc}{\ensuremath{{\rm c}}}

\newcommand{\xii}{\ensuremath{x_{\rm i}}}
\newcommand{\ximo}{\ensuremath{x_{\rm i-1}}}
\newcommand{\xipo}{\ensuremath{x_{\rm i+1}}}
\newcommand{\fii}{\ensuremath{f_{\rm i}}}
\newcommand{\fimo}{\ensuremath{f_{\rm i-1}}}
\newcommand{\fipo}{\ensuremath{f_{\rm i+1}}}
\newcommand{\dxi}{\ensuremath{\Delta x_{\rm i}}}
\newcommand{\dxipo}{\ensuremath{\Delta x_{\rm i+1}}}


\newcommand{\chinu}{\ensuremath{\chi_{\nu}}}
\newcommand{\etanu}{\ensuremath{\eta_{\nu}}}
\newcommand{\Inu}{\ensuremath{I_{\nu}}}
\newcommand{\Snu}{\ensuremath{S_{\nu}}}
\newcommand{\Hnu}{\ensuremath{H_{\nu}}}
\newcommand{\Jnu}{\ensuremath{J_{\rm \nu}}}
\newcommand{\Bnu}{\ensuremath{B_{\rm \nu}}}

\newcommand{\eqrt}{\hbox{EQRT}}

\newcommand{\malf}{\ensuremath{M_{\rm A}}}
\newcommand{\valf}{\ensuremath{v_{\rm A}}}
\newcommand{\ralf}{\ensuremath{R_{\rm A}}}
\newcommand{\rapex}{\ensuremath{r_{\rm m}}}
\newcommand{\rkep}{\ensuremath{R_{\rm K}}}
\newcommand{\Bpole}{\ensuremath{B_{\rm p}}}
\newcommand{\etastar}{\ensuremath{\eta_{\rm \ast}}}
\newcommand{\mdotz}{\ensuremath{\dot{M}_{\rm B=0}}}
\newcommand{\mdotb}{\ensuremath{\dot{M}_{\rm B}}}
\newcommand{\vthfid}{\ensuremath{v_{\rm th}^{\rm \ast}}}
\newcommand{\ddopfid}{\ensuremath{\Delta \nu_{\rm D}^{\rm \ast}}}
\newcommand{\ddop}{\ensuremath{\Delta \nu_{\rm D}}}

\newcommand{\wrt}{w.r.t.}
\newcommand{\sigmae}{\ensuremath{\sigma_{\rm e}}}
\newcommand{\sigmab}{\ensuremath{\sigma_{\rm B}}}
\newcommand{\kboltz}{\ensuremath{k_{\rm B}}}
\newcommand{\nel}{\ensuremath{n_{\rm e}}}


\newcommand{\mbh}{\ensuremath{M_{\rm BH}}}

\newcommand{\Rpole}{\ensuremath{R_{\rm p}}}
\newcommand{\Req}{\ensuremath{R_{\rm eq}}}
\newcommand{\Teffeq}{\ensuremath{T_{\rm eff, eq}}}
\newcommand{\Teffpole}{\ensuremath{T_{\rm eff, p}}}

%\newcommand{\clatitude}{\ensuremath{\tilde{\theta}}}
\newcommand{\clatitude}{\ensuremath{\Theta}}
\newcommand{\azimuth}{\ensuremath{\Phi}}
%
%radiation coordinates
\newcommand{\thetarad}{\ensuremath{\theta}}
\newcommand{\phirad}{\ensuremath{\phi}}

%how to write stellar names from HD catalogue
\newcommand{\hdname}{\textit{}}
\newcommand{\omp}{\textsc{OpenMP}}

%transition rates
\newcommand{\Cul}{\ensuremath{C_{\rm ul}}}
\newcommand{\Aul}{\ensuremath{A_{\rm ul}}}

%corrections
\newcommand{\corr}[1]{\textcolor{red}{#1}}
\newcommand{\todo}[1]{\textcolor{red}{#1}}

%monotonicity parameter
\newcommand{\paramon}{\ensuremath{\omega}}

%mean relative error
\newcommand{\mrerr}[1]{\mkern 1.5mu\overline{\mkern-1.5mu#1\mkern-1.5mu}\mkern 1.5mu}
%\newcommand{\mrerr}[1]{\ensuremath{<#1>}}

%number of iterations until convergence
\newcommand{\nconv}{\ensuremath{N_{\rm iter}}}

%units
\newcommand{\kK}{\ensuremath{{\rm kK}}}
\newcommand{\kelvin}{\ensuremath{{\rm K}}}
\newcommand{\yr}{\ensuremath{{\rm yr}}}
\newcommand{\Gauss}{\ensuremath{{\rm G}}}
\newcommand{\kGauss}{\ensuremath{{\rm kG}}}

%electronmass
\newcommand{\mel}{\ensuremath{m_e}}

%square degree
\newcommand{\degsqr}{\ensuremath{{\rm deg}^2}}

%convolution
\newcommand{\convolve}{\ensuremath{\circledast}}


\def\aj{AJ}%
          % Astronomical Journal
\def\araa{ARA\&A}%
          % Annual Review of Astron and Astrophys
\def\apj{ApJ}%
          % Astrophysical Journal
\def\apjl{ApJ}%
          % Astrophysical Journal, Letters
\def\apjs{ApJS}%
          % Astrophysical Journal, Supplement
\def\ao{Appl.~Opt.}%
          % Applied Optics
\def\apss{Ap\&SS}%
          % Astrophysics and Space Science
\def\aap{A\&A}%
          % Astronomy and Astrophysics
\def\aapacc{A\&A}%
          % Astronomy and Astrophysics (accepted / in publication)
\def\aa{A\&A}%
          % Astronomy and Astrophysics
\def\aapr{A\&A~Rev.}%
          % Astronomy and Astrophysics Reviews
\def\aaps{A\&AS}%
          % Astronomy and Astrophysics, Supplement
\def\azh{AZh}%
          % Astronomicheskii Zhurnal
\def\baas{BAAS}%
          % Bulletin of the AAS
\def\jrasc{JRASC}%
          % Journal of the RAS of Canada
\def\memras{MmRAS}%
          % Memoirs of the RAS
\def\mnras{MNRAS}%
          % Monthly Notices of the RAS
\def\pra{Phys.~Rev.~A}%
          % Physical Review A: General Physics
\def\prb{Phys.~Rev.~B}%
          % Physical Review B: Solid State
\def\prc{Phys.~Rev.~C}%
          % Physical Review C
\def\prd{Phys.~Rev.~D}%
          % Physical Review D
\def\pre{Phys.~Rev.~E}%
          % Physical Review E
\def\prl{Phys.~Rev.~Lett.}%
          % Physical Review Letters
\def\pasp{PASP}%
          % Publications of the ASP
\def\pasj{PASJ}%
          % Publications of the ASJ
\def\qjras{QJRAS}%
          % Quarterly Journal of the RAS
\def\skytel{S\&T}%
          % Sky and Telescope
\def\solphys{Sol.~Phys.}%
          % Solar Physics
\def\sovast{Soviet~Ast.}%
          % Soviet Astronomy
\def\ssr{Space~Sci.~Rev.}%
          % Space Science Reviews
\def\zap{ZAp}%
         % Zeitschrift fuer Astrophysik
\def\na{New~Astron.}%
          % New Astronomy
\def\nat{Nature}%
          % Nature
\def\iaucirc{IAU~Circ.}%
          % IAU Cirulars
\def\aplett{Astrophys.~Lett.}%
          % Astrophysics Letters
\def\apspr{Astrophys.~Space~Phys.~Res.}%
          % Astrophysics Space Physics Research
\def\bain{Bull.~Astron.~Inst.~Netherlands}%
          % Bulletin Astronomical Institute of the Netherlands
\def\fcp{Fund.~Cosmic~Phys.}%
          % Fundamental Cosmic Physics
\def\gca{Geochim.~Cosmochim.~Acta}%
          % Geochimica Cosmochimica Acta
\def\grl{Geophys.~Res.~Lett.}%
          % Geophysics Research Letters
\def\jcp{J.~Chem.~Phys.}%
          % Journal of Chemical Physics
\def\jgr{J.~Geophys.~Res.}%
          % Journal of Geophysics Research
\def\jqsrt{J.~Quant.~Spec.~Radiat.~Transf.}%
          % Journal of Quantitiative Spectroscopy and Radiative Trasfer
\def\memsai{Mem.~Soc.~Astron.~Italiana}%
          % Mem. Societa Astronomica Italiana
\def\nphysa{Nucl.~Phys.~A}%
          % Nuclear Physics A
\def\physrep{Phys.~Rep.}%
          % Physics Reports
\def\physscr{Phys.~Scr}%
          % Physica Scripta
\def\planss{Planet.~Space~Sci.}%
          % Planetary Space Science
\def\procspie{Proc.~SPIE}%
          % Proceedings of the SPIE
\let\astap=\aap
\let\apjlett=\apjl
\let\apjsupp=\apjs
\let\applopt=\ao

%
%-----------------------------------------------------------------------
%
\title{\notesTitle}
\author{\notesAuthor}
%
%-----------------------------------------------------------------------
%
\begin{document}
\maketitle
%
\clearpage
\tableofcontents
\clearpage
%
\section{Introduction}
The aim of this code library is to perform radiative transfer
calculations for a range of applications, focussing however
on the spectral synthesis of 3D hot-star winds. The main
features consist of performing detailed radiative transfer
calculations accounting for highly supersonic velocity fields and
(almost) arbitrary 3D structures. We have developed several main
programs to be used for different situations, summarized as follows:
%
\begin{description}
  \item[line3D:] A code module to calculate synthetic line profiles for a single star within a global \textit{star-in-a-box} setup (see Sect.~\ref{subsec:line3d}).
  \item[BOSS-3D:] A code module to calculate synthetic line profiles for binary systems (see Sect.~\ref{subsec:cont3d}).  
  \item[cont3Dslab:] A code module to calculate the continuum radiation for a single star within a local \textit{star-in-a-box} setup, that might be used for coupling in radiation-hydrodynamic simulations (see Sect.~\ref{subsec:boss3d}).
\end{description}
%
Each of these packages will be described in the corresponding sections in more detail.

\subsection{Radiation hydrodynamics}
ToDo

\subsection{Radiative transfer}
To calculate the radiative transfer, we consider the time-independent equation of radiative transfer,
%
\beq
\label{eq:eqrt}
\vecown{n}\vecown{\nabla} I_\nu = \eta_\nu - \chi_\nu I_\nu = \chi_\nu \left(S_\nu - I_\nu\right)\,,
\eeq
%
with $I_\nu$ the specific intensity, $\eta_\nu$ the emissivity,
$\chi_\nu$ the opacity, and $S_\nu=\eta_\nu/\chi_\nu$ the source
function.  Further, we define the angular moments of the specific
intensity:
%
\beqa
   J_\nu &=& \dfrac{1}{4 \pi} \int I_\nu \dd \Omega = \dfrac{c}{4\pi} E_\nu \,,\\
   \vecown{H}_\nu &=&  \dfrac{1}{4\pi} \int I_\nu \vecown{n} \dd \Omega = \dfrac{1}{4\pi} \vecown{F}_\nu  \,, \\
   \matown{K}_\nu &=&  \dfrac{1}{4\pi} \int \underbrace{\vecown{n} I_\nu \vecown{n}}_{{\rm dyadic~product}} \dd \Omega = \dfrac{c}{4\pi}\matown{P}_\nu \,,
\eeqa
%   
with $J_\nu$ the mean intensity, $\vecown{H}_\nu$ the Eddington flux,
and $K_\nu$ simply the second moment without a specific name.  The
mean intensity, Eddington flux, and second moment are trivially
related to the radiation energy density $E_\nu$, the radiation flux
$\vecown{F}_\nu$, and the radiation pressure tensor $\matown{P}_\nu$.

For all code modules, we are typically considering the radiation quantities
$I_\nu$, $J_\nu$, $\vecown{H}_\nu$, $\matown{K}_\nu$, and not the
corresponding `physical' quantities. In general there are three
operating modes for calculating these quantities:

\begin{itemize}
\item[(i)] Since the source function in the equation of radiative
  transfer (EQRT) can in principle depend on the radiation field,
  Eq.~\eqref{eq:eqrt} becomes an integro-differential equation. In
  this case, an iteration scheme is required (main code
  \textit{line3D/sc3d.eo} and \textit{cont3Dslab/sc3d.eo} for
  star-in-a-box and box-in-a-star simulations, respectively) based on
  a non-local accelerated $\Lambda$ iteration (ALI). 
\item[(ii)] For known source functions and opacities (\eg~approximated
  in LTE or pre-calculated in step (i)), we can solve the radiative
  transfer in a $pz$-type geometry to obtain surface brightnesses or
  emergent flux profiles (main codes \textit{line3D/modelspec.eo} and
  \textit{line3D/spec.eo}).
\item[(iii)] If we are dealing with binary systems, we rely purely on
  semi-analytical models thus far (\eg~opacities and source functions
  in LTE), and solve the radiative transfer in a $pz$-type of geometry
  (main codes \textit{line3D/modelspec\_vbin.eo} and
  \textit{line3D/spec\_vbin.eo}).
\end{itemize}
%
%
%
\subsection{Philosophy}
All developed sub-programs are meant to be -- at least in
principle -- a sort of stand-alone packages, that can be used
completely independent of each other. Indeed, we consider the full
radiative transfer problem as a three-step process:
\begin{enumerate}
\item Firstly, we need to create a discretized model of the physical
  state of the gas (\ie~density $\rho$, gas temperature $\Tgas$, velocity
  field $\vecown{v}$). In order that our radiative-transfer routines
  can communicate with such a model, we have developed a user
  interface to either transform input data from a given hydrodynamic
  simulation or to set up a semi-analytical model. This step
  is performed in the code \textit{model.eo} with corresponding source
  code to be found in \textit{line3D/src\_model} and
  \textit{cont3Dslab/src\_model}.
  \item Secondly, we need to calculate opacities and source functions. We can follow two branches here:
    \begin{itemize}
    \item For resonance lines within a two-level-approximation and/or a two-component continuum source consisting of thermal and scattering terms, we can
      calculate continuum and line source functions consistently with
      the radiation field. This is performed by the code
      \textit{sc3d.eo}. Since the iteration scheme is computationally
      very expensive, the source function will be calculated on a
      relatively low-resolution grid, and then interpolated back onto the original model within the code \textit{modelspec.eo}.
      The corresponding source codes can be found in
      \textit{line3D/src\_sc3d}, \textit{line3D/src\_modelspec}, and
      \textit{cont3Dslab/src\_sc3d}.
      \item Alternatively, we can directly use the code
        \textit{modelspec.eo} to calculate source functions and
        opacities from semi-analytical calculations (\eg~assuming LTE occupation numbers, see \textit{line3D/src\_modelspec}).
    \end{itemize}
  \item Finally, the surface brightness or emergent flux profiles are
    calculated by solving the radiative transfer in a $pz$-type
    geometry using the code \textit{spec.eo}, with input given from
    the previous step 2. The corresponding source code can be found in
    \textit{line3D/src\_spec} and \textit{cont3Dslab/src\_surfb}.
\end{enumerate}
%
We emphasize that the binary version (extension \textit{\_vbin})
consists only of steps (2) and (3), since we haven't implemented an
ALI scheme for binary systems yet.

\section{Installation}
\subsection{Requirements}
The code requires the following packages:
\begin{description}
\item[\textsc{FORTRAN} compiler:] Either the gfortran (version  8+) or ifort compiler is required. Depending which compiler is used, one needs to adapt the source code since the \textsc{inquire} function works differently for both compilers:
  \\
  gfortran: inquire(file=trim(directory)//'/.', exist=my\_boolean)
  \\
  ifort: inquire(directory=trim(directory), exist=my\_boolean)
\item[\textsc{HDF5}:] The HDF5 library is required (version 1.10.5 or higher). This library needs to be compiled with the same compiler as used for the main programs (\ie~gfortran or ifort). See Sect.~\ref{subsec:hdf5} on how to install HDF5 on your system
  \item[\textsc{PYTHON, IDL, GDL}] The code comes with an IDL/GDL or PYTHON library for reading and plotting all output files.
\end{description}  
\subsection{Getting the code}
You can get the code from github:
\begin{itemize}
\item Global star-in-a-box simulations:\\
  git clone https://github.com/levin-h/line3D
\item Local box-in-a-star simulations:\\
  git clone https://github.com/levin-h/cont3D
\end{itemize}
%
%
%
\subsection{Quick start}


\subsection{Installing HDF5}
\label{subsec:hdf5}
I recommend to create a local build for your HDF5 libraries. To this end, please follow the following steps (here for version 1.10.6)
\paragraph{UNIX systems}
\begin{enumerate}
\item Download the package \textbf{hdf5-1.10.6.tar.gz}, unzip it, and change to the corresponding folder: \\
  tar -zxvf hdf5-1.10.6.tar.gz\\
  cd hdf5-1.10.6
\item Export some required environment variables:\\
    \begin{tabular}{ll}
      \textbf{gfortran}  & \textbf{ifort} \\
      export FC=gfortran & export FC=ifort \\
      export CC=gcc & export CC=icc \\
      export F9X=gfortran & export F9X=ifort \\
      export CXX=g++ & export CXX=icpc
    \end{tabular}
\item Configure your installation with a local path where you want to install the package (\eg~\$HOME/hdf5\_gfortran\_build):\\
  ./configure -{}-prefix=\$HOME/hdf5\_gfortran\_build -{}-enable-fortran (and if required -{}-enable-cxx)
\item Installation \\
  make (and watch for fatal errors)\\
  make check   (and verify that all tests return a `pass') \\
  make install
\item Include the library path in the Makefile of the main code (\eg~line3D/Makefile)
\item Update your .bashrc file to add the library path to your LD environment variable:\\
    HDF5\_PATH=\$HOME/hdf5\_gfortran\_build/hdf5\_lib/lib \\
    LD\_LIBRARY\_PATH=\$LD\_LIBRARY\_PATH:\$ADD\_LIB\_PATH \\
    export LD\_LIBRARY\_PATH
\end{enumerate}

\paragraph{MAC}
On the MAC, the installation is essentially performed the same way. For the gfortran compiler, you might need
\begin{enumerate}
\item Install the CommandLineTools
\item Install homebrew:\\
  mkdir homebrew \&\& curl -L https://github.com/Homebrew/brew/tarball/master | tar xz -{}-strip 1 -C homebrew \\
  eval "\$(homebrew/bin/brew shellenv)" \\
  brew update -{}-force -{}-quiet \\
  chmod -R go-w "\$(brew -{}-prefix)/share/zsh"  \\
  export PATH=\$HOME/homebrew/bin:\$PATH
\item Install gfortran:\\
  brew install gcc
\item Install HDF5 as for UNIX systems. When updating the Makefile of
  the main code (\eg~line3D/Makefile), replace all *.so libraries with
  the MAC *.dylib extension.
\end{enumerate}

\subsection{Importing the plotting routines}
All plotting routines are stored in the directory
\textit{line3D/plotFILES} or \textit{cont3Dslab/plotFILES}, with
corresponding libraries named \textit{levpy}, \textit{lib\_gdl} and
\textit{lib\_idl} for \textsc{PYTHON}, \textsc{GDL}, and \textsc{IDL},
respectively.
%
\paragraph{GDL/IDL}
The required libraries can be simply read in by typing in your IDL/GDL terminal:\\
@idl\_lib/startup.pro \\
@gdl\_lib/startup.pro \\

We emphasize that some of the routines might be outdated (for
instance, variable names stored within the individual *.h5 files might
have changed). It should be straight forward though to adapt the
corresponding reading routines (in \textit{idl\_lib/proLEV/getall}).
%
\paragraph{PYTHON}
The following python packages are required: \\
h5py, imageio, matplotlib, numpy, os, scipy, sys
%
%
%
\section{Main code modules}
\subsection{line3D}
\label{subsec:line3d}
This folder contains code modules for running global (star-in-a-box)
radiative transfer simulations. There are essentially four different
code modules further described in the following. Each code module
requires a specific input file organized by namelists.

\begin{description}
\item[model.eo] Prepares the data to be used for the actual radiative
  transfer calculations, by transforming any input data or calculating
  semi-analytic models. You can simply add new models in
  src\_model. The corresponding namelist file is described in
  Table~\ref{tab:namelist_sc3d}. Essentially, we set up the state of
  the gas in 1d, 2d, or 3d, described by the density $\rho$, the
  velocity field $\vecown{v}$, the gas and radiation temperatures,
  $\Tgas$ and $\Trad$, the thermal velocities (becomes obsolete at some
  point), $\vth$, and the thermalization parameter $\epsc$. Currently hardcoded, the model will be saved in
  inputFILES/modelXd.h5.
\item[sc3d.eo] Performs the radiative transfer for the continuum
  and/or line transition with certain opacity laws using an iterative
  ALI scheme. The corresponding namelist file is described in
  Table~\ref{tab:namelist_sc3d}.
\item[modelspec.eo] Prepares the data to be
  used for the line-profile calculations of single stars. Here, we can
  read in the source functions and opacities calculated by the
  \textit{sc3d.eo} program, or implement other semi-analytic
  models. The corresponding namelist file is described in
  Table~\ref{tab:namelist_modelspec}.
\item[spec.eo] Calculates line profiles for a specific input file. The
  corresponding namelist files is described in
  Table~\ref{tab:namelist_spec}.
\end{description}

\subsubsection{Program model.eo}
All source files for this program are stored in the directory
\textit{src\_model}, and the corresponding namelist file is summarized
in Table~\ref{tab:namelist_sc3d}. Within this namelist, there are many
input parameters that are actually not required. For consistency
reasons and to avoid potential error sources, we decided to use the
same namelist file also for the program \textit{sc3d.eo}.

To register a new model, we recommend to follow the following steps:
\begin{enumerate}
\item In \textit{src\_model/model.f90}, add a new model identifier as
  a case for the variable input\_mod. This identifier should be used
  also in the namelist file to call this particular model.
  The subroutine to create the model still needs to be developed by
  the user, \eg~\textit{calc\_my\_model}, and needs to be called within the case
  of the new model identifier. Further, depending on the
  dimensionality of the new model, we need to save it as an h5 file by
  calling the (already existing) subroutines \textit{output\_mod*d}.
  
\item In \textit{src\_model/model*d.f90}, we create our new subroutine
  \textit{calc\_my\_model}. Depending on the dimension of the new
  model, different global variables have to be set (see
  \textit{src\_model/output\_model.f90} for more details). For a 3D
  model in spherical coordinates $(r,\clatitude,\azimuth)$, we would require the following:
  \begin{description}
  \item[nr\_modext] describes the number of radial grid points for our model.
  \item[ntheta\_modext] describes the number of $\clatitude$ grid points for our model.
  \item[nphi\_modext] describes the number of $\azimuth$ grid points for our model.
  \item[r\_modext3d] describes the radial grid (array of length nr\_modext) in cgs.
  \item[theta\_modext3d] describes the $\clatitude$ grid (array of length ntheta\_modext) from $[0,\pi]$.
  \item[phi\_modext3d] describes the $\azimuth$ grid (array of length nphi\_modext) from $[0,2\pi]$.
  \item[velr\_modext3d] describes the radial velocity component (array of length nr\_modext, ntheta\_modext, nphi\_modext) in cgs.
  \item[velth\_modext3d] describes the $\clatitude$ velocity component (array of length nr\_modext, ntheta\_modext, nphi\_modext) in cgs.
  \item[velphi\_modext3d] describes the $\azimuth$ velocity component (array of length nr\_modext, ntheta\_modext, nphi\_modext) in cgs.
  \item[rho\_modext3d] describes the density (array of length nr\_modext, ntheta\_modext, nphi\_modext) in cgs.
  \item[t\_modext3d] describes the gas temperature (array of length nr\_modext, ntheta\_modext, nphi\_modext) in cgs.
  \item[trad\_modext3d] describes the radiation temperature (array of length nr\_modext, ntheta\_modext, nphi\_modext) in cgs. Often used only as a dummy array.
  \item[vth\_modext3d] describes the thermal velocity (array of length nr\_modext, ntheta\_modext, nphi\_modext) in cgs. Often used only as a dummy array.
  \item[eps\_cont\_modext3d] describes the thermalization parameter (array of length nr\_modext, ntheta\_modext, nphi\_modext).
    \end{description}
\end{enumerate}
%
To plot the resulting model, you can use the programs
\textit{plotFILES/model*d.py} or \textit{plotFILES/model*d.pro} for
PYTHON or IDL/GDL, respectively.
%
\begin{footnotesize}
\begin{longtable}[h]{p{0.24\linewidth}p{0.07\linewidth}p{0.69\linewidth}}
\caption{\normalsize Input namelist for the programs \textit{model.eo} and
  \textit{sc3d.eo}. Some of the inputs are meanwhile obsolete. We use one indat file for both programs to avoid inconsistency of the data used within \textit{model.eo} and \textit{sc3d.eo}. If only running the \textit{model.eo} many of the paramneters are not required and should be assigned with an arbitrary value.}
\label{tab:namelist_sc3d}
\\\hline\hline
 Example & Data type & Description \\\hline
  \&input\_options & & Options for the models \\
model\_dir = 'inputFILES' & string & Directory of the model that will be read in \\
output\_file = 'output\_model00.h5' & string & All calculations stored in output\_file (.h5 extension to be included) \\
input\_mod = 12 & integer & Identifier of the model to be calculated; only required for model.eo (where the hydro model is specified)\\
input\_mod\_dim = 3 & integer & Dimension of input model; input\_mod\_dim $\in [1,2,3]$ \\
%
spatial\_grid1d = 5 & integer & Identifier to calculate a 1D radial grid from a beta-velocity law. Depending ong the option spatial\_grid3d, this input is obsolete.\\
 & & spatial\_grid1d=0 if equidistant radial grid is used (subroutine grid1d\_r\_equi) \\
 & & spatial\_grid1d=1 if equidistant velocity grid is used (subroutine grid1d\_vel\_equi) \\
 & & spatial\_grid1d=2 if equidistant tau\_thomson grid is used (subroutine grid1d\_tau\_equi) \\
 & & spatial\_grid1d=3 if equidistant log(tau\_thomson) grid is used (subroutine grid1d\_tau\_log) \\
 & & spatial\_grid1d=4 if combination is used (see subroutine grid1d\_final for details) \\
 & & spatial\_grid1d=5 if combination is used (see subroutine grid1d\_final\_2 for details) \\
 & & spatial\_grid1d=6 if grid is calucalted equidistant in log-space (subroutine grid1d\_r\_log) \\
%
spatial\_grid3d = 2 & integer & Identifier to calculate the 3D Cartesian grid. \\
& & spatial\_grid3d=0 if 3d grid is calculated from 1d grid with equidistant core points  \\
& & spatial\_grid3d=1 if 3d grid is calculated from a mean-value approach (minimizing distance of subsequent coordinates from 1d-grid) \\
& & spatial\_grid3d=2 if 3d grid is calculated from a mean-value approach  (minimizing distance of subsequent coordinates from original input-grid) \\
& & spatial\_grid3d=3 if 3d grid is calculated completely equidistant \\
& & spatial\_grid3d=4 if 3d grid is calculated from a 1d radial grid and setting up angular grid equidistantly \\
& & spatial\_grid3d=5 if 3d grid is calculated from a 3d spherical grid (optimized) \\
%
opt\_opac = 0 & integer & Identifier to decide on the continuum opacity model \\
 & & opt\_opac=0 if Thomson opacities \\
 & & opt\_opac=1 if OPAL opacities \\
%
opt\_opal = 0 & integer & Identifier to decide on the line opacity model \\
 & & opt\_opal=0 if line-strength parameter \\
 & & opt\_opal=1 if \cite{Hamann80} parameterization \\
%
opt\_angint\_method = 9 & integer & Identifier to decide on the angular-integration technique to be used \\
 & & opt\_angint\_method=0 if angular integration is used with trapezoidal rule (nodes equidistant in $\theta$ and $\phi$) \\
 & & opt\_angint\_method=1 if angular integration is used with trapezoidal rule (nodes from \cite{Lobel08}) \\
 & &  opt\_angint\_method=2 if angular integration is used with simpsons rule (nodes equidistant in $\theta$ and $\phi$, note: $\mu$-grid and $\phi$-grid will be made equidistant for three subsequent points) \\
 & & opt\_angint\_method=3 if angular integration is used with simpson rule corrected for the error from a grid with half resolution (also known as boole's rule) \\
 & & opt\_angint\_method=4 if angular integration is used with cubic splines (catmull-rom-spline, nodes equidistant in $\theta$ and $\phi$) \\
 & & opt\_angint\_method=5 if angular integration is used with gauss-legendre-integration (for each octant) \\
 & & opt\_angint\_method=6 if angular integration is used with gauss-chebyshev-integration (for each octant) \\
 & & opt\_angint\_method=7 if angular integration is used with triangulation (linear integrals) \\
 & & opt\_angint\_method=8 if angular integration is used with triangulation ('pseudo'-gauss integrals per triangle) \\
 & & opt\_angint\_method=9 if angular integration is used with lebedev interpolation (optimized nodes on the sphere) \\
%
opt\_method = 1 & integer & Identifier to decide on the radiative-transfer solution method \\
 & & opt\_method=0  if finite volume method shall be used \\
 & & opt\_method=1  if linear short characteristics method shall be used \\
 & & opt\_method=2  if quadratic bezier short characteristics method shall be used \\
%
opt\_sol2d = f & logical & Logical to decide whether 2D solution scheme shall be applied \\
%
opt\_ltec = 0 & integer & Identifier to decide on the continuum wavelength/frequency model \\
 & & opt\_ltec = 0 if single continuum frequency \\
& & opt\_ltec = 1 if grey approximation for continuum (frequency
integrated). If this option is set, the temperature will be updated
after the radiation-transfer calculations assuming radiative equilibrium (\ie~from $J=S=B=\sigmab/\pi \,T^4$). \\
%
opt\_incl\_cont = t & logical & Set to true (false) if continuum shall be included (or not) \\
opt\_start\_cont = t & logical  & Set to true (false) if continuum iteration shall start from the beginning (or from intermediate steps) \\
opt\_ng\_cont = t & logical  & Set to true (false) if Ng-extrapolation for continuum iteration shall be included or not \\
opt\_ait\_cont = f & logical & Set to true (false) if Aitkens-extrapolation for continuum iteration shall be included or not \\
opt\_incl\_line = f & logical & Set to true (false) if line shall be included (or not) \\
opt\_start\_line = t & logical & Set to true (false) if line iteration shall start from the beginning (or from intermediate steps) \\
opt\_ng\_line = t & logical  & Set to true (false) if Ng-extrapolation for line iteration shall be included or not \\
opt\_ait\_line = f & logical & Set to true (false) if Aitkens-extrapolation for line iteration shall be included or not \\
opt\_alo\_cont = 3 & integer & Identifier to define the approximate $\Lambda$-operator for continuum iteration \\
 & & opt\_alo\_cont = 0 if classical $\Lambda$ iteration \\
 & & opt\_alo\_cont = 1 if diagonal approximate $\Lambda$ operator \\
 & & opt\_alo\_cont = 2 if direct-neighbour approximate $\Lambda$ operator (7 elements) \\
 & & opt\_alo\_cont = 3 if nearest-neighbour approximate $\Lambda$ operator (27 elements) \\
opt\_alo\_line = 3 & integer & Identifier to define the approximate $\Lambda$-operator for line iteration \\
 & & opt\_alo\_line = 0 if classical $\Lambda$ iteration \\
 & & opt\_alo\_line = 1 if diagonal approximate $\Lambda$ operator \\
 & & opt\_alo\_line = 2 if direct-neighbour approximate $\Lambda$ operator (7 elements) \\
 & & opt\_alo\_line = 3 if nearest-neighbour approximate $\Lambda$ operator (27 elements) \\
opt\_incl\_gdark = f & logical & Set to true (false) if gravity darkening by \cite{Zeipel24} shall be included (or not) \\
opt\_incl\_sdist = f & logical & Set to true (false) if surface distortion due to rotation shall be included (or not) \\\hline
%
\&input\_mod\_1d & & Input parameters of the considered star (some not required anymore) \\
%
teff = 40.d3 & float & Effective temperature of the star in [K] \\
trad = 40.d3 & float & Radiation temperature of the star (used as the inner boundary condition for the specific intensity) in [K] \\
xlogg = 3.5d0 & float & $\logg$ of the star \\
rstar = 8.d0 & float & $\Rstar$ in $\rsun$ \\
lstar = 1.d6 & float & $\Lstar$ in $\lsun$ \\
rmax = 12.d0 & float & Maximum radius of the computational domain in $[\Rstar]$ along each $x,y,z$ axis \\
tmin = .8d0 & float & Minimum temperature in the wind in [\Trad] \\
xmloss = 5.d-6 & float & mass-loss rate \mdot~in \msunyr; only required for 1D benchmarking \\
vmin = 1.d1 & float & minimum velocity of $\beta$-velocity law \vmin~in \kms; only required for 1D benchmarking \\
vmax = 2.d3 & float & terminal velocity of $\beta$-velocity law \vinf~in \kms; only required for 1D benchmarking \\
vmicro = 1.d2 & float & micro-turbulent velocity for the line-profile function \vturb~in [\kms] \\
vth\_fiducial= 1.d2 & float & fiducial thermal velocity \vthfid~in [\kms] \\
vrot = 0.d0 & float & rotational velocity \vrot~in [\kms] \\
beta = 1.d0 & float & $\beta$ parameter for $\beta$-velocity law; only required for 1D benchmarking models \\
yhe = .1d0 & float & Helium abundance by number, $\yhe$ \\
hei = 2.d0  & float & Helium ionization fraction (number of free electrons per Helium-atom) \\
xnue0 = 1.93798d15 & float & Frequency of the line transition \\
na = 12 & integer & mass number $A$ for the line transition \\\hline
\&input\_infreg & & Input parameters to define the computational domain (information region) \\
rmin = 1.d0 & float & Minimum radius of the computational domain in \Rstar \\
rlim = 13.2d0 & float & Maximum radius of the computational domain in \Rstar \\\hline
%
\&input\_cont & & Parameters for the continuum transport \\
eps\_cont = 0.d0 & float & Thermalization parameter \epsc\\
kcont = 1.d0 & float & \kcont~parameter (linear scaling factor for the continuum opacity) \\
%
\&input\_line & & Parameters of the line transport \\
eps\_line = 0.d0 & float & Line-scattering parameter \epsl \\
kline = 1.d0 & float & line-strength parameter \kline \\
kappa0 = 1.d-1 & float & \cite{Hamann80} parameterization \\
alpha = 0.5d0  & float & \cite{Hamann80} parameterization \\\hline
%
\&dimensions\_1d & & Dimension parameters to set up 1D radial grid \\
n1d = 17 & integer & number of radial grid points (used to distribute $z$-axis in [\Rmin,\Rmax]) \\
n1d\_t = 81 & integer & number of 1D grid points to set up equidistant $\tau$-grid \\
n1d\_r = 22 & integer & number of 1D grid points to set up equidsitant $v_r$-grid \\
delv = 0.33d0 & float & Preferred velocity steps $\Delta v_r$ in \vthfid \\\hline
%
\&dimensions\_3d & & Dimension parameters to set up the 3D grid \\
ncx=19 & integer & Preferred number of core-points for $x$-axis \\
ncy=19 & integer & Preferred number of core-points for $y$-axis \\
ncz=19 & integer & Preferred number of core-points for $z$-axis \\
delx\_max=.7d0  & float & Maximum allowed $\Delta x$ in \Rstar \\
dely\_max=.7d0 & float & Maximum allowed $\Delta y$ in \Rstar \\
delz\_max=.7d0 & float & Maximum allowed $\Delta z$ in \Rstar \\\hline
%
\&dimensions\_freq & & Dimension parameters to set up the frequency grid \\
deltax = 0.333d0 & float & $\Delta \xobs$ steps \\
xcmf\_max = 3.d0 & float & Maximum frequency width of the line-profile function,  $\xcmf^{\rm (max)}$  \\\hline
%
\&dimensions\_angles & & Dimension parameters to set up the angular grid \\
n\_theta = 11 & integer & Number of $\theta$ angles in first octant; $\phi$ angles are calculated based on that \\\hline
%
\&benchmark & & Parameters for setting up a benchmark \\
benchmark\_mod = 0 & integer & Identifier to define the benchmark model (set to 0 if no benchmark shall be performed) \\
im\_source = 3  & integer & see benchmark subroutines \\
im\_opacity = 2 & integer & see benchmark subroutines \\
im\_vel = 0 & integer & see benchmark subroutines \\
tau\_min = 0.d0 & float & see benchmark subroutines \\
tau\_max = 5.d0 & float & see benchmark subroutines \\
source\_min = 0.1d0 & float & see benchmark subroutines \\
source\_max = 1.d-6 & float & see benchmark subroutines \\
n\_y = 0.d0 & float & see benchmark subroutines \\
n\_z = 0.707107d0 & float & see benchmark subroutines
\end{longtable}
\end{footnotesize}

\subsubsection{Program sc3d.eo}
This program solves the non-linear coupling of the radiative transfer equation with the source function of the form:
\beqa
\scont &=& \left(1 - \epsc \right) \Jnu + \epsc \Bnu \\
\sline &=& \left(1 - \epsl \right) \Jbar + \epsl B_{\nu_0} \,,
\eeqa
\ie~for a continuum consisting of thermal and scattering terms, and
for a resonance-line transition approximated as a two-level atom.
To this end, we are discretizing the equation of radiative transfer in
Cartesian coordinates, and rely on the accelerated $\Lambda$-iteration (ALI)
using non-local approximate $\Lambda$ operators (ALO). The
corresponding source files can be found in \textit{src\_sc3d}.

There are various different methods for solving the radiative transfer
equation (\eg~via the finite-volume method or the
short-characteristics method), as well as for performing the
source-function updates (using different ALO's).
All available options required for the input namelist are summarized
in Table~\ref{tab:namelist_sc3d}. As output and depending on the
chosen options, the *.h5 file generated by \textit{sc3d.eo} provides
among other data:
\begin{description}
\item[scont3d] The continuum source function \scont~in cgs (3d array with dimensions (nx,ny,nz)).
\item[mint3d] The mean intensity \Jnu~in cgs (3d array with dimensions (nx,ny,nz)).
\item[fcontx3d, fconty3d, fcontz3d] The Eddington flux components in Cartesian coordinates, $\vecown{H}_\nu = (H_x, H_y, H_z)$, in cgs (3d array with dimensions (nx,ny,nz)).
\item[kcontxx3d, kcontyy3d, kcontzz3d, kcontxy3d, kcontxz3d,
  kcontyz3d] The tensor components of the $\matown{K}_\nu$-tensor (3d arrays with dimensions (nx,ny,nz)). We
  emphasize that this is a symmetric tensor, and only six components need to be saved to deduce the complete tensor.
\item[mintbar3d] The frequency integrated and profile weighted mean intensity, \Jbar~in cgs (3d array with dimensions (nx,ny,nz)).
\item[sline3d] The line source function, \sline~in cgs (3d array with dimensions (nx,ny,nz)).
\end{description}

To plot the resulting model, we provide the PYTHON and GDL/IDL
programs \textit{plotFILES/plot\_sc3d.py} and
\textit{plotFILES/plot\_sc3d.pro}.
%

\subsubsection{Program modelspec.eo}
This program prepares data to be used for calculating spectral
features and/or surface brightnesses of our simulations.
We can either calculate a line profile from the output (\ie~source
functions) of the \textit{sc3d.eo} program, or create a completely new
semi-analytic model. All source files can be found in the
\textit{src\_modelspec/} directory, with the available namelist options summarized in Table~\ref{tab:namelist_modelspec}.

To create a new model here, we essentially follow the same philosophy
as for the \textit{model.eo} program, and recommend the following two steps:
\begin{itemize}
\item In \textit{src\_modelspec/modelspec.f90}, we can add a new model identifier as a case for the namelist variable inuput\_mod.
  As before, we can then create and call a new subroutine describing our model.
  \item In \textit{src\_modelspec/modelspec.f90}, we also create the
    new subroutine, \eg~\textit{subroutine my\_model}. Within this
    subroutine (or in the input namelist), we need to specify the following global variables (here for a
    standard 3D model in spherical coordinates).
    \begin{description}
    \item[nr] Number of radial grid points.
    \item[ntheta] Number of $\clatitude$ grid points.
    \item[nphi] Number of $\azimuth$ grid points.
    \item[r] The radial grid in \Rstar.
    \item[theta] The $\clatitude$ grid in the range $[0,\pi]$.
    \item[phi] The $\phi$ grid in the range $[0,2\pi]$.
    \item[sline3d] The line source function in cgs (3d array with dimensions (nr,ntheta,nphi)).
    \item[scont3d] The continuum source function in cgs (3d array with dimensions (nr,ntheta,nphi)).
    \item[t3d] The gas temperature in cgs (3d array with dimensions (nr,ntheta,nphi)).
    \item[opac3d] The continuum opacity in $[1/\Rstar]$ (3d array with dimensions (nr,ntheta,nphi)).
    \item[oplbar3d] The frequency integrated line opacity in $[1/s\Rstar]$ (3d array with dimensions (nr,ntheta,nphi)).
    \item[velx3d] The x-component of the velocity field in cgs (3d array with dimensions (nr,ntheta,nphi)).
    \item[vely3d] The y-component of the velocity field in cgs (3d array with dimensions (nr,ntheta,nphi)).
    \item[velz3d] The z-component of the velocity field in cgs (3d array with dimensions (nr,ntheta,nphi)).
    \item[xic1, xic2] The anchor for the inner boundary condition of the specific intensity, which should follow the form for core rays:
      \beq
      \Inu = xic1\cdot q_1 - xic2 \cdot q_2 \,,
      \eeq
      where $q_1$ and $q_2$ are scaling factors to be calculated during the formal solution (\eg~$q_1$ can be set to account for gravity darkening). A reasonable choice, for instance might be:
      \beq
      xic1 = \Bnu\left(\Teff\right) \qquad \qquad xic2 = \dfrac{\dd B\nu}{\chi_\nu \dd z} \,.
      \eeq
      \end{description}
  \end{itemize}
%
Again, we can display the resulting model by using the programs
\textit{plotFILES/modelspec3d.py} or
\textit{plotFILES/modelspec3d.pro}.
%
%
\begin{footnotesize}
\begin{longtable}[h]{p{0.24\linewidth}p{0.07\linewidth}p{0.69\linewidth}}
\caption{\normalsize Input namelist for the program \textit{modelspec.eo}}
\label{tab:namelist_modelspec}
\\\hline\hline
 Example & Data type & Description \\\hline
\&input\_options & & Main options \\
\specialcell[t]{input\_file = \\'./outputFILES/output\_model00.h5'} &
 string & Name of the input file generated by \textit{sc3d.eo}, if source funtions and opacities are to be read in from the solution of \textit{sc3d.eo} \\
\specialcell[t]{input\_file2 = \\ './inputFILES/model3d.h5'} & string & Name of the input model file generated by \textit{model.eo}. Depending on the input\_mod options, all opacities and source  functions are either interpolated from the \textit{sc3d.eo} output onto this grid, or calculated from a semi-analytical model. This procedure allows us to use a low-resolution grid for the computationally challenging ALI iteration, while still using a high-resolution grid of the wind's density and velocity structure. \\
\specialcell[t]{output\_file = \\ './outputFILES/modspec\_model00.h5'} & string & Output file \\
input\_mod = 19 & integer & Identifier for the model to be calculated (see in ./src\_modelspec/modelspec.f90). There are a few standard options to communicate with the output from the program \textit{sc3d.eo}, such as:\\
& & input\_mod=11  3d model: standard ouput from sc3c.eo (3d cartesian model) \\
& & input\_mod=12  3d model: standard ouput from sc3c.eo (3d cartesian model) interpolated onto the spherical grid from the model.eo output \\

\\\hline
%
\&input\_model & & Parameters of the input model \\
teff = 258390.7d0 & float & Effective temperature of the star. Only required to get the correct photospheric line profile later on. \\
trad = 258390.7d0 & float & Radiation temperature of the star. Only used to set the inner boundary condition for the specific intensity. \\
xlogg = 3.6d0 & float & $\logg$ of the star. Only used to get the correct photospheric line profile later on. \\
rstar = 1.d0 & float & $\Rstar$ in $\rsun$ \\
rmax = 11.d0 & float & $\Rmax$ in $\Rstar$, used to define the computational domain \\
tmin = 1.d0  & float & Minimum temperature of the wind in [$\Teff$]. Only used for very specific test routines. \\
xmloss = 1.d-6 & float & Mass-loss rate $\mdot$ in [\msunyr]. Only used for very specific test routines. \\
vmin = 10.d0  & float & Minimum velocity \vmin~of a $\beta$-velocity law in [\kms]. Only used for very specific test routines \\
vmax = 4.d3  & float & Terminal velocity \vinf~of a $\beta$-velocity law in [\kms]. \textbf{If not overwritten within the specific model routines, this sets also the range of velocities/frequencies for which the line-profiles are calculated} \\
beta = 1.d0  & float & $\beta$ parameter of a $\beta$-velocity law in [\kms]. Only used for very specific test routines \\
vmicro = 1.0d2  & float & Microturbulent velocity \vturb in [\kms]. \\
vth\_fiducial=1.d2 & float & Fiducial thermal velocity to be used in  [\kms].\\ 
yhe = 0.1d0 & float & Helium number abundance, $\yhe=\nhe/\nh$ (\eg~$\yhe=12.25$ corresponds to mass-fraction 0.98). \\
hei = 2.d0  & float & Number of free electrons per helium atom \\\hline
%
\&input\_line & & Line parameters \\
iline = 0 & integer & Identifier for the line (as defined in src/mod\_iline.f90) to get all line data ($\nu_0, g_l, g_u$, etc) \\
& & iline=0 - read atomic charge $Z$, element $i$, lower level $l$ and upper level $u$ from file 'in\_linelist.dat' \\
& & iline=1 - \Ha \\
& & iline=2 - \Hb \\
& & iline=10 - C\,IV resonance line \\
& & iline=11 - C\,III\,5696 line \\
eps\_line = 0.d0 & float & Line scattering parameter \epsl. Only used for specific test routines (Sobolev solution) \\
kline = 1.d0 & float & Line-strength parameter or arbitrary scaling factor to increase/decrease the line opacity \\
kappa0 = 1.d0 & float & \cite{Hamann80} parameterization \\
alpha = 0.d0 & float & \cite{Hamann80} parameterization \\\hline
%
\end{longtable}
\end{footnotesize}


\subsubsection{Program spec.eo}
This program calculates synthetic line profiles and surface
brigthnesses for a given model obtained by the program
\textit{modelspec.eo}. To this end, we rely on a cylindric coordiante
system $(p,\zeta,z)$ (see also \cite{Hennicker2021}).
When calculating surface brightnesses the output will be stored as
*.h5 file giving:
\begin{description}
\item[p] The array of impact parameters.
\item[zeta] The array of angles of the cylindrical coordinate system
  \item[iem\_surface] The (total) emergent intensity at each
    $p,\zeta$ in cgs.
  \item[iemi\_surface] The emission part of the total intensity at each
    $p,\zeta$ in cgs.
  \item[iabs\_surface] The absorption part of the total intensity at each
    $p,\zeta$ in cgs.
  \item[icont\_surface] The continuum intensity only (if there was no line) at each
    $p,\zeta$ in cgs.
\end{description}

When calculating emergent flux profiles, the output will be stored as ASCII files in \textit{FLUXEM\_*.dat}.
The output is organized in columns giving:
\begin{description}
\item[xobs] The frequency shift from line center in units of the fiducial velocity \vthfid.
\item[flux\_tot] The total emergent flux-like (or rather
  luminosity-like) quantity at this frequency. Following,
  \eg~\citet[Sect. 3.7]{Hennicker2020}, the flux is given by:
  \beq
  \Fnu = \dfrac{1}{d^2} \underbrace{\int_0^{2\pi} \int_0^{\Rmax} \Inu\left(p, \zeta, z=\Rmax \right) p \dd p \dd \zeta}_{=: \textrm{flux\_tot}}\,.
  \eeq
  Since we have been integrating over the impact parameter $p$ (which internally is measured in \Rstar), we can translate the quantity flux\_tot to a luminosity in cgs:
  \beq
  \Lnu = \textrm{flux\_tot} \cdot \Rstar^2 \cdot 4 \pi\,.
  \eeq

\end{description}



The namelist options for the program \textit{spec.eo} are summarized in Table~\ref{tab:namelist_spec}.
%
\begin{footnotesize}
\begin{longtable}[h]{p{0.24\linewidth}p{0.07\linewidth}p{0.69\linewidth}}
\caption{\normalsize Input namelist for the program \textit{spec.eo}}
\label{tab:namelist_spec}
\\\hline\hline
 Example & Data type & Description \\\hline
\&input\_options & & Main options \\
 input\_mod = 2 & integer & Type of the input model \\
 & & input\_mod = 0 -- 1D model on radial grid \\
 & & input\_mod = 1 -- 3D model on Cartesian grid \\
 & & input\_mod = 2 -- 3D model on spherical grid \\
 \specialcell[t]{input\_file = \\ './outputFILES/modspec\_model00.h5'} &
 string & Name of the input file generated by \textit{modelspec.eo} \\
\specialcell[t]{output\_dir = \\ './outputFILES'} & string & Output directory \\
opt\_photprof = 0 & integer & Identifier for defining the photospheric line profile \\
& & opt\_photprof = 0 - no photospheric line profile (flat illumination) \\
& & opt\_photprof = 1 - from A. Herrero files \\
& & opt\_photprof = 2 - from Kurucz (not active at the moment) \\
& & opt\_photprof = 3 - from own FASTWIND compilation (only active in the binary version at the moment) \\
& & opt\_photprof = 4 - from \cite{Coelho2005} (only active in the binary version at the moment) \\
& & opt\_photprof = 5 - from \cite{Coelho2014} (only active in the binary version at the moment) \\
opt\_obsdir\_read = t & logical & Logical to decide whether observer's direction shall be read in or calculated. \\
& & opt\_obsdir\_read = t -- read in angles $\alpha\in[0,180]$ (measured from the $z$-axis, inclination) and $\gamma\in[0,360]$ (measured from the $x$-axis, phase) from files in\_alpha.dat and in\_gamma.dat \\
& & opt\_obsdir\_read = f -- Equidistant $\alpha$, $\gamma$ grid will be calculated based on input options nalpha and ngamma. \\
opt\_surface = t & logical & Logical to decide if surface brightness shall be calculated instead of emergent flux profiles. \\
opt\_int2d = f & logical & Logical to decide if the propagation of intensity along a 2D slice trough the computational domain shall be calculated instead of emergent flux profiles \\
opt\_incl\_gdark = f & logical & Logical to decide if \cite{Zeipel24} gravity darkening shall be included \\
opt\_incl\_sdist = f & logical & Logical to decide if surface distortion shall be accounted for \\
nalpha = 1 & integer & Number of $\alpha$ angles to define the directions to the observer \\
ngamma = 1 & integter & Number of $\gamma$ angles to define the directions to the observer \\\hline
%
\&input\_model & & Input parameters for the model \\
vrot = 0.d0 & float & Surface rotation of the star in [\kms] (at the equator). \\
vth\_fiducial = 1.d2 & float & Fiducial thermal velocity \vthfid~in[\kms].\\
vmicro = 1.0d2 & float & Microturbulent velocity \vturb~in[\kms].\\
rmin = 1.d0 & float & Minimum radius of the computational domain (as used for \textit{modelspec.eo}).\\
rmax = 10.97d0 & float & Maximum radius of the computational domain
(typically a bit smaller than used for \textit{modelspec.eo} to avoid
extrapolation errors/interpolations to zero). \\\hline
%
\&input\_surface & & Input parameters for surface brightness calculations and calculating intensities along a 2d slice. Will be used only if either
opt\_surface or opt\_int2d is set to true \\
nsurfb = 2 & integer & Number of surface brightnesses to be calculated. \\
\specialcell[t]{alpha\_surface = \\ 1.570796d0, 1.570796d0} & float & The $\alpha$ angles towards the observer (number of elements needs to be equal to nsurfb). \\
\specialcell[t]{gamma\_surface = \\ 0.d0, 0.d0} & float & The $\gamma$ angles towards the observer (number of elements needs to be equal to nsurfb). \\
\specialcell[t]{xobs\_surface = \\  0.d0, 10.d0} & float & The shift from line center in units of $\vthfid$ (number of elements needs to be equal to nsurfb)\\
& & For this example, two surface brightnesses will be calculated with directions and frequencies taken from (i) the first elements of the arrays and (ii) the second elements of the arrays. 
%
\end{longtable}
\end{footnotesize}
%
%
%
%
\subsection{boss3D}
\label{subsec:boss3d}
The BOSS-3D package (also within the \textit{line3D} folder with
corresponding programs \textit{line3D/modelspec\_vbin.eo} and
\textit{line3D/spec\_vbin.eo}, the extension \textit{vbin}
abbreviating `version binary') contains modules for running global
(star-in-a-box) radiative transfer simulations of binary
systems. There are two different code modules further described in the
following. Each code module requires a specific input file organized
by namelists.
%
\begin{description}
\item[modelspec\_vbin.eo] Prepares the data to be used for the
  line-profile calculations of binary systems. Here, we define the
  model in terms of density, temperature and velocity fields, as well
  as opacities and source functions. The corresponding namelist file
  is described in Table \ref{tab:namelist_modelspec_vbin}.
\item[spec\_vbin.eo] Calculates line profiles for a specific input
  file. The corresponding namelist file is described in
  Table~\ref{tab:namelist_spec_vbin}.
\end{description}
%
\subsubsection{The namelist files}
The namelist options for the programs \textit{modelspec\_vbin.eo}, and
\textit{spec\_vbin.eo} are shown in Tables
\ref{tab:namelist_modelspec_vbin} and \ref{tab:namelist_spec_vbin},
respectively.
\begin{footnotesize}
\begin{longtable}[h]{p{0.24\linewidth}p{0.07\linewidth}p{0.69\linewidth}}
\caption{\normalsize Input namelist for the program \textit{modelspec\_vbin.eo}, \ie~for the binary version.}
\label{tab:namelist_modelspec_vbin}
\\\hline\hline
 Example & Data type & Description \\\hline
\&input\_options & & Main options \\
input\_file = '' & string & Name of the input file generated by \textit{sc3d.eo} (not used yet in binary version) \\
input\_file2 = '' & string & Name of the input model file generated by \textit{model.eo} (not used yet in binary version) \\
\specialcell[t]{output\_file = \\ './outputFILES/modspec\_model00.h5'} & string & Output file \\
input\_mod = 9 & integer & Identifier for the model to be calculated (see in ./src\_modelspec\_vbin/modelspec.f90). \\\hline
%
\&input\_model1 & & Parameters of the input model for the primary object. \\
rstar1 = 1.d0 & float & $\Rstar^{(1)}$ of the primary object in $\rsun$, defining the length scale of the coordinate system of the primary. \\
rmin1 = 1.d0 & float & Minimum radius defining the computational domain of the primary object, $\Rmin$ in $\Rstar^{(1)}$ \\
rmax1 = 10.d0 & float & Maximum radius defining the computational domain of the primary object, $\Rmax$ in $\Rstar^{(1)}$ \\
teff1 = 6.d3 & float & Effective temperature of the primary object. Only required to get the correct photospheric line profile later on. \\
trad1 = 6.d3 & float & Radiation temperature of the primary object. Only used to set the inner boundary condition for the specific intensity. \\
logg1 = 1.d0 & float & $\logg$ of the primary object. Only used to get the correct photospheric line profile later on. \\
yhe1 = 0.1d0 & float & Helium number abundance, $\yhe=\nhe/\nh$, of the primary object. \\
fehe1 = -1.d0 & float & Fe/He abundance of primary object. Only required for \cite{Coelho2005} and \cite{Coelho2014} photospheric line profiles. \\
aenh1 = 0.d0 & float & $\alpha$-element enhancement of primary object. Only required for \cite{Coelho2005} and \cite{Coelho2014} photospheric line profiles. \\
vrot1 = 10.d0 & float & Surface rotation of the primary object in [\kms] (at its equator). \\
vmicro1 = 1.0d2  & float & Microturbulent velocity of the primary object \vturb in [\kms]. \\
\specialcell[t]{p\_object01 = \\
  0.d0, 3.d0, 0.d0} & float & $x,y,z$ position of the primary object within the global (center-of-mass) coordinate system in units [unit\_length] (see \&input\_units) \\
\specialcell[t]{v\_object01 = \\
  -10.d0, 0.d0, 0.d0} & float & $v_x, v_y, v_z$ components of the orbit of the primary object within the global center-of-mass coordinate system in [\kms].\\
ex01 = 1.d0, 0.d0, 0.d0 & float & Orientation of the $\vecown{e}_x$ basis vector of the primary object within the global center-of-mass coordinate system. \\
ey01 = 0.d0, 1.d0, 0.d0 & float & Orientation of the $\vecown{e}_y$ basis vector of the primary object within the global center-of-mass coordinate system. \\
ez01 = 0.d0, 0.d0, 1.d0 & float & Orientation of the $\vecown{e}_z$ basis vector of the primary object within the global center-of-mass coordinate system. \\
\specialcell[t]{rot\_axis01  = \\ 0.d0, 0.d0, 1.d0} & float & Orientation of the rotation axis of the primary object within the global center-of-mass coordinate system (still to be implemented).
\\\hline
\&input\_model2 & & Parameters of the input model for the secondary object. Same as for primary object but interchaning the variable name index 1 with 2. \\
rstar2 = 3.d0 & float & $\Rstar^{(2)}$ of the secondary object in $\rsun$, defining the length scale of the coordinate system of the secondary. \\
rmin2 = 1.d0 & float & Minimum radius defining the computational domain of the secondary object, $\Rmin$ in $\Rstar^{(2)}$ \\
rmax2 = 100.d0 & float & Maximum radius defining the computational domain of the secondary object, $\Rmax$ in $\Rstar^{(2)}$ \\
teff2 = 10.d3 & float & Effective temperature of the secondary object. Only required to get the correct photospheric line profile later on. \\
trad2 = 10.d3 & float & Radiation temperature of the secondary object. Only used to set the inner boundary condition for the specific intensity. \\
logg2 = 3.d0 & float & $\logg$ of the secondary object. Only used to get the correct photospheric line profile later on. \\
yhe2 = 0.1d0 & float & Helium number abundance, $\yhe=\nhe/\nh$, of the secondary object. \\
fehe2 = -1.d0 & float & Fe/He abundance of secondary object. Only required for \cite{Coelho2005} and \cite{Coelho2014} photospheric line profiles. \\
aenh2 = 0.d0 & float & $\alpha$-element enhancement of secondary object. Only required for \cite{Coelho2005} and \cite{Coelho2014} photospheric line profiles. \\
vrot2 = 100.d0 & float & Surface rotation of the secondary object in [\kms] (at its equator). \\
vmicro2 = 1.0d1  & float & Microturbulent velocity of the secondary object \vturb in [\kms]. \\
\specialcell[t]{p\_object02 = \\
  0.d0, -2.d0, 0.d0} & float & $x,y,z$ position of the secondary object within the global (center-of-mass) coordinate system in units [unit\_length] (see \&input\_units) \\
\specialcell[t]{v\_object02 = \\
  6.d0, 0.d0, 0.d0} & float & $v_x, v_y, v_z$ components of the orbit of the secondary object within the global center-of-mass coordinate system in [\kms].\\
ex02 = 1.d0, 0.d0, 0.d0 & float & Orientation of the $\vecown{e}_x$ basis vector of the secondary object within the global center-of-mass coordinate system. \\
ey02 = 0.d0, 1.d0, 0.d0 & float & Orientation of the $\vecown{e}_y$ basis vector of the secondary object within the global center-of-mass coordinate system. \\
ez02 = 0.d0, 0.d0, 1.d0 & float & Orientation of the $\vecown{e}_z$ basis vector of the secondary object within the global center-of-mass coordinate system. \\
\specialcell[t]{rot\_axis01  = \\ 0.d0, 0.d0, 1.d0} & float & Orientation of the rotation axis of the secondary object within the global center-of-mass coordinate system (still to be implemented).\\\hline
%
\&input\_line & & Line parameters \\
iline = 0 & integer & Identifier for the line (as defined in src/mod\_iline.f90) to get all line data ($\nu_0, g_l, g_u$, etc) \\
& & iline=0 - read atomic charge $Z$, element $i$, lower level $l$ and upper level $u$ from file 'in\_linelist.dat' \\
& & iline=1 - \Ha \\
& & iline=2 - \Hb \\
& & iline=10 - C\,IV resonance line \\
& & iline=11 - C\,III\,5696 line \\
eps\_line = 0.d0 & float & Line scattering parameter \epsl. Only used for specific test routines (Sobolev solution) \\
kline = 1.d0 & float & Line strength parameter \\\hline
%
\&input\_units & & Units of the simulation \\
unit\_length = 1.d0 & float & Length scale of the global coordinate system in [\rsun] \\
vth\_fiducial = 1.0d2 & float & Fiducial thermal velocity \vthfid. \\\hline
%
\end{longtable}
\end{footnotesize}
%
%
%
\begin{footnotesize}
\begin{longtable}[h]{p{0.24\linewidth}p{0.07\linewidth}p{0.69\linewidth}}
\caption{\normalsize Input namelist for the program \textit{spec\_vbin.eo}, \ie~for the binary version}
\label{tab:namelist_spec_vbin}
\\\hline\hline
Example & Data type & Description \\\hline
\&input\_options & & Main options \\
 input\_mod = 2 & integer & Type of the input model \\
 & & input\_mod = 2 -- 3D model on spherical grid \\
 \specialcell[t]{input\_file = \\ './outputFILES/modspec\_model00.h5'} &
 string & Name of the input file generated by \textit{modelspec\_vbin.eo} \\
\specialcell[t]{output\_dir = \\ './outputFILES'} & string & Output directory \\
opt\_photprof1 = 5 & integer & Identifier for defining the photospheric line profile of the primary object. \\
opt\_photprof2 = 0 & integer & Identifier for defining the photospheric line profile of the secondary object. \\
& & opt\_photprof = 0 - no photospheric line profile (flat illumination) \\
& & opt\_photprof = 1 - from A. Herrero files \\
& & opt\_photprof = 2 - from Kurucz (not active at the moment) \\
& & opt\_photprof = 3 - from own FASTWIND compilation \\
& & opt\_photprof = 4 - from \cite{Coelho2005} \\
& & opt\_photprof = 5 - from \cite{Coelho2014} \\
opt\_obsdir\_read = t & logical & Logical to decide whether observer's direction shall be read in or calculated. \\
& & opt\_obsdir\_read = t -- read in angles $\alpha\in[0,180]$ (measured from the $z$-axis of the global center-of-mass coordinate system, inclination) and $\gamma\in[0,360]$ (measured from the $x$-axis of the global center-of-mass coordinate system, phase angle) from files in\_alpha.dat and in\_gamma.dat \\
& & opt\_obsdir\_read = f -- Equidistant $\alpha$, $\gamma$ grid will be calculated based on input options nalpha and ngamma. \\
opt\_surface = t & logical & Logical to decide if surface brightness shall be calculated instead of emergent flux profiles. \\
opt\_int2d = f & logical & Logical to decide if the propagation of intensity along a 2D slice trough the computational domain shall be calculated instead of emergent flux profiles \\
opt\_incl\_gdark1 = f & logical & Logical to decide if \cite{Zeipel24} gravity darkening shall be included for primary object \\
opt\_incl\_sdist1 = f & logical & Logical to decide if surface distortion of primary object shall be accounted for \\
opt\_incl\_gdark2 = f & logical & Logical to decide if \cite{Zeipel24} gravity darkening shall be included for secondary object \\
opt\_incl\_sdist2 = f & logical & Logical to decide if surface distortion of secondary object shall be accounted for \\
opt\_pgrid01 = 'log' & string & Defining the $p$-grid stratification of the primary object. \\
opt\_rgrid01 = 'log' & string & Defining the $r$-grid stratification of the primary object. \\
opt\_pgrid02 = 'lin' & string & Defining the $p$-grid stratification of the secondary object. \\
opt\_rgrid02 = 'lin' & string & Defining the $r$-grid stratification of the secondary object. \\
& & 'lin' -- linear stratification \\
& & 'log' -- logarithmic stratification \\
& & 'llog' -- $\log-\log$ stratification \\

nalpha = 1 & integer & Number of $\alpha$ angles to define the directions to the observer \\
ngamma = 1 & integter & Number of $\gamma$ angles to define the directions to the observer \\\hline
%
\&input\_model & & Input parameters for the model \\
vth\_fiducial = 1.d2 & float & Fiducial thermal velocity \vthfid~in[\kms].\\\hline
%
\&input\_surface & & Input parameters for surface brightness calculations and calculating intensities along a 2d slice. Will be used only if either
opt\_surface or opt\_int2d is set to true \\
alpha\_surface = 1.570796d0 & float & The $\alpha$ angle towards the observer. \\
gamma\_surface = 0.d0 & float & The $\gamma$ angle towards the observer. \\
xobs\_surface = 0.d0 & float & The shift from line center in units of $\vthfid$ \\
& & Note: In contrast to the single-star version, we here only allow for one surface brightness to be calculated at a time. 
%
\end{longtable}
\end{footnotesize}




%
\subsection{cont3Dslab}
\label{subsec:cont3d}
%


\section{Getting started}
\section{Related papers}


\section{Cite}
Depending on the code modules you are using, we would kindly ask you to cite one of the following papers:
\begin{itemize}
\item For the ALI scheme using the finite-volume method, please cite \cite{Hennicker2018}.
\item For the ALI scheme using the short-characteristics method, please cite \cite{Hennicker2020}.
\item For the formal solution calculating emergent flux profiles or surface brightnesses of single stars, please cite \cite{Hennicker2018} and/or \cite{Hennicker2021}.
\item For the formal solution calculating emergent flux profiles or surface brightnesses of binary systems, please cite \cite{Hennicker2021}.
\item For LTE tabulations of occupation numbers, please cite \cite{Poniatowski2021} (ToDo Luka).
\end{itemize}
Thank you very much. \\
For further reading on the ALI method, we refer to \cite{Hennicker2020b}.


\section{Developers and contributors}
These code modules have been developed in collaboration with: N. Moens, L. Poniatowski., J. Puls, S. Sundqvist.
The radiative transfer modules use parts of the GEOMPACK2 library\footnote{\url{https://people.math.sc.edu/Burkardt/f_src/geompack2/geompack2.html}} and
EISPACK libraries\footnote{\url{https://people.sc.fsu.edu/~jburkardt/f77_src/eispack/eispack.html}} (see \cite{Joe1991}).

\section{Known problems and solutions}
Below, you can find a list of known problems and (possible) solutions:
\begin{description}
\item[OMP is not working:] There are (at least) two possibilities that can cause
  these problems: OMP\_FLAG is not set in the Makefile (set it to
  -fopenmp). Alternatively, you might not have set the enivironment
  variable on your system (in the terminal, simply use export OMP\_NUM\_THREADS=N, with N the number of OMP threads to be used, \eg, 12).
\item[Segmentation fault when running in OMP mode:] By default, the
  stacksize for each thread can be very low. Particularly when
  requiring huge arrays in the formal solution with a lot of
  grid refinement (\eg for LDI simulations), the local copies in each thread might run out of stacksize. Probably at the expense of computing efficiency, this problem can be solved by setting the corresponding environment variable: export OMP\_STACKSIZE=10M  (or larger if required).
\item[Comiplation errors of HDF5] Sometimes, a new fortran compiler is
  not compatible with an old HDF5 version. Then you might want to
  switch to a later HDF5 release (version 1.10.7 or higher), or
  downgrade your fortran compiler.

\item[Mac -- Illegal Instruction 4:] On the Mac, an Illegal Instruction 4 error can occur when static arrays are not properly
  initialized. To solve this issue, and to still be able to use \textsc{OpenMP} parallelization, please dynamically allocate static arrays, e.g.:\\
  real(dp), dimension(nd) :: my\_array \\
  becomes \\
  real(dp), dimension(:), allocatable :: my\_array \\
  allocate(my\_array(nd))
\end{description}


\section{ToDo}
\begin{itemize}
\item In src/mod\_iline.f90, LTE tables are read in only for 'lte\_tables/Y02800'
\end{itemize}


\section{Acknowledgements}
We like to thank B. Joe and J. Burkardt for providing the GEOMPACK2 libraries.
Further, we thank the developers of the EISPACK library.

The development of all code modules has been funded by the German
Research Foundation, DFG, under grant PU 117/9-1, and by the Odysseus
program of the Belgian Research Foundation Flanders (FWO) under grant
G0H9218N.

%\begin{figure}[ht]
%\resizebox{\hsize}{!}{
%   \begin{minipage}{0.5\hsize}
%      \resizebox{\hsize}{!}{\includegraphics{plots/paperI/mhd_model.ps}}
%   \end{minipage}
%   \begin{minipage}{0.5\hsize}
%      \resizebox{\hsize}{!}{\includegraphics{plots/paperI/mhd_model_average.ps}}
%   \end{minipage}
%}
%%file: testsDATABASE/mhd_plots/mhd.pro
%%
%\caption[Small caption.]{Long caption}
%%
%\label{fig:density_mhd}
%\end{figure}




\bibliographystyle{aamod}
\bibliography{bib_levin}

%\begin{thebibliography}{test}
%   \bibitem[label printout]{biblabel} author, n. : ''title'', AA,  2004;
%\end{thebibliography}


\end{document}

